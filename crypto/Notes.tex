\documentclass{report}
\usepackage{amsmath,array,hyperref,amssymb,mathrsfs,listings,physics}
\setlength{\parindent}{0pt}
\hypersetup{colorlinks,citecolor=black,filecolor=black,linkcolor=black,urlcolor=black}
\newcommand{\tmod}{\text{mod }}
\newtheorem{theorem}{Theorem}[section]
\newtheorem{corollary}{Corollary}[theorem]
\newtheorem{lemma}[theorem]{Lemma}
\lstset{language=python}
\title{Cryptography and attacks}
\author{Ariana}
\date{\today}
\begin{document}
\maketitle
\tableofcontents{}
\part{Mathematical preliminaries}
Since computers aren't the best at storing arbitrary reals, we usually use integers, which is how \textbf{number theory} gets involved. Encryption/Hashing is basically a function that maps integers to integers, in a way that is hard to reverse.

Furthermore, having extremely large integers would make computing extremely long, so we use finite groups to avoid having overly large numbers, which involves some \textbf{abstract algebra}.

Nowadays, elliptic curves cryptography and variants are getting quite popular. The theory behind elliptic curves have only been formulated in the mid-1900, largely from the work of the mathematician Grothendieck, who effectively kick-started \textbf{algebraic geometry}. This topic is rather math heavy so try not to have high hopes on understanding the topic in a few years(it's usually taught in math grad school for reference). 

As quantum computing is getting more powerful and more accurate nowadays, quantum-resistant cryptography are also getting more important. This new form of cryptography encompasses many different ideas, from classical cryptography(not using quantum properties) such as lattices to isogeny, to quantum cryptography that uses several interesting properties of quantum systems that classical systems cannot replicate.

\section{Notation}
\begin{itemize}
	\item[]$S\\{a}\}$ - The set $S$ without the element/set $a$
	\item[]$\mathbb{N}$ - The set of natural numbers, including $0$
	\item[]$\mathbb{Z}$ - The set of integers
	\item[]$\mathbb{Z}^+$ - The set of positive integers
	\item[]$\mathbb{Z}^-$ - The set of negative integers
	\item[]$\mathbb{Z}_p^+$ - The additive group modulo $p$
	\item[]$\mathbb{Z}_p^\times$ - The multiplicative group modulo $p$
	\item[]$\mathbb{P}$ - The set of prime numbers
	\item[]$a|b$ - $b$ is divisible by $a$, $\exists m^{\in\mathbb{Z}}b=ma$
	\item[]$a\perp b$ - $a$ and $b$ are coprime, equivalently, $GCD(a,b)=1$
\end{itemize}
\chapter{Number theory}
Number theory is a branch of mathematics dedicated to studying integers and equations involving integer solutions.
\section{Divisibility and primes}
The notion of divisibility and remainder appears, often implicitly, everywhere in cryptography. This section introduces the basic notion of divisibility and remainder.

\begin{theorem}
	Division theorem
	
	Given some $a,b$, with $b>0$, there is a unique solution to $a=qb+r$, where $0\leq r<|b|$.
\end{theorem}
This is quite simple to proof, and the uniqueness is proven by contradiction.

The \textbf{greatest common divisor(GCD)} of $a$ and $b$ is the largest natural number that divides $a$ and $b$. $q$ is called the \textbf{quotient} and $r$ is called the \textbf{remainder}.

A extremely simple algorithm, coming all the way from Euclid, is the \textbf{Euclidean algorithm}. The algorithm has a simple recursive definition:

$$r_{-2}=a\quad r_{-1}=b$$
$$r_{k-2}=q_kr_{k-1}+r_k\quad\text{from the division theorem}$$
This stops when $r_N=0$, and $r_{N-1}$ is the GCD.

A worked example for $a=1113$ and $b=812$ is given below(this case was specially chosen, the algorithm usually converges a lot faster):

\begin{tabular}{|c|c|c|}\hline
	k&eqn&$q$ and $r$\\\hline
	-2&&$r_{-2}=1113$\\\hline
	-1&&$r_{-1}=812$\\\hline
	0&$1113=812q_0+r_0$&$q_0=1,r_0=301$\\\hline
	1&$812=301q_1+r_1$&$q_1=2,r_1=210$\\\hline
	2&$301=210q_2+r_2$&$q_2=1,r_2=91$\\\hline
	3&$210=91q_3+r_3$&$q_3=2,r_3=28$\\\hline
	4&$91=28q_4+r_4$&$q_4=3,r_4=7$\\\hline
	5&$28=7q_5+r_5$&$q_5=4,r_5=0$\\\hline
\end{tabular}

Thus the GCD is $7$.

By reversing this algorithm, we also get solutions to a linear diophantine equation.

\begin{theorem}
	B\'ezout's identity
	
	Given some $a,b$, there is exactly one solution to $xa+yb=\gcd(a,b)=d,|x|\leq|\frac{b}{d}|,|y|\leq|\frac{a}{d}|$. Other solutions to this equations are of the form $\left(x-k\frac{b}{d},y+k\frac{a}{d}\right)$
\end{theorem}
The proof of existence is simple, by solving for $x,y$ using the extended euclidean algorithm. $(x,y)$ are usually referred to as the \textbf{B\'ezout's coefficients} Using $a=1113$ and $b=812$:

\begin{tabular}{|c|c|c|}\hline
	k&eqn&$q$ and $r$\\\hline
	0&$1113=812q_0+r_0$&$q_0=1,r_0=301$\\\hline
	1&$812=301q_1+r_1$&$q_1=2,r_1=210$\\\hline
	2&$301=210q_2+r_2$&$q_2=1,r_2=91$\\\hline
	3&$210=91q_3+r_3$&$q_3=2,r_3=28$\\\hline
	4&$91=28q_4+r_4$&$q_4=3,r_4=7$\\\hline
	5&$28=7q_5+r_5$&$q_5=4,r_5=0$\\\hline
\end{tabular}

Now we `reverse' the algorithm by starting off with eqn 4, then continuously substitute in the previous equations.
\begin{align*}
7&=91-28\cdot3\\
7&=91-\left(210-91\cdot2\right)\cdot3\\
7&=91\cdot7-210\cdot3\\
7&=(301-210\cdot1)\cdot7-210\cdot3\\
7&=301\cdot7-210\cdot10\\
7&=301\cdot7-(812-301\cdot2)\cdot10\\
7&=(1113-812\cdot1)\cdot27-812\cdot10\\
7&=1113\cdot27-812\cdot37\\
\end{align*}
which are the B\'ezout's coefficients$(27,-37)$

Another key aspect of number theory is the concept of \textbf{prime numbers}. A prime number is simply a number that is only divisible by $1$ and itself. The first few primes are:
$$2,3,5,7,11,13,\dots$$
This may seem pretty simple but primes appear very often, especially in analytic number theory(if anyone wants to have some fun there).

Numbers that are not prime are called \textbf{composite numbers}. These numbers are a product of primes(if they aren't they will be primes itself). We call the representation of a number as a product of primes the \textbf{prime factorization}. This comes with quite a obvious but important theorem:

\begin{theorem}
	Prime factorization is unique
\end{theorem}

A important fact about primes is that for any composite number, it's prime factorization is unique. This can easily be proven by contradiction and using the fact primes are unique.

Now for a general composite number, suppose we want to count the number of integers below and coprime to itself. This function is known as the Euler's totient function($\phi(n)$). It is pretty tedious to count one at a time, however, with the prime factorization, there's a formula that relies on prime factorization(Euler's product formula):

$$\phi(n)=n\prod_{p|n}\left(1-\frac{1}{p}\right)$$

The proof of this requires $2$ easier to proof facts:
\begin{enumerate}
	\item If $m\perp n$, then $\phi(mn)=\phi(m)\phi(n)$
	\item For any prime $p$, $\phi(p^k)=p^k\left(1-\frac{1}{p}\right)$
\end{enumerate}
The first part is proven by showing that if $a\perp m$ and $b\perp n$, then $an+bm\perp mn$, and that the other direction holds too.

The second part is done by finding all the numbers that aren't coprime to $p^k$.

Finally using the fact prime factorization is unique, the Euler's product formula is proven.
\section{Modular arithmetic}

\emph{The proofs in this section requires some group theory, so its best to just look through the basics, and revisit this section later on.}

Usually, we are not so interested in the entire number, but just the remainder after being divided by some number. This is the idea of modular arithmetic, i.e. after every operation, we take the remainder.

For example in $\mod5$:
$$1+2\equiv3$$
$$1-2\equiv4$$
$$4\cdot2\equiv3$$
$$\frac{3}{4}\equiv2$$

Notice that division is also possible. However, this may not always be possible. The idea of modular division is given $\frac{a}{b}$, we want to find a number $c$ such that $a\equiv bc$, similar to normal division. In fact, with B\'ezout's identity, this is rather simple to solve.

$$\frac{1}{b}\equiv c\pmod n$$
$$bc\equiv 1\pmod n$$
$$cb+kn=1$$
So when $\gcd(b,n)=1$, this is always solvable, with $(c,k)$ as the B\'ezout's coefficients. $c$ is known as the \textbf{modular multiplicative inverse}

In modular arithmetic, modular exponents are quite interesting, with a lot of theorems associated with it, due to its structure as a abelian group. For example,

$$5^6\equiv5\pmod11$$

Now, if you were given $5^x\equiv5\pmod11$, this becomes quite intimidating rather quickly, since our notion of the real numbered log is gone. This is known as the \textbf{discrete log problem}. Fortunately there are some theorems that can assist in solving this problem. 

\begin{theorem}
	Wilson's theorem
	
	$$(n-1)!\equiv-1\pmod n\Leftrightarrow n\in\mathbb{P}$$
\end{theorem}

The forward implication is quite simple, if $n\notin\mathbb{P}$, there exists $a$ integer $a$ such that $a<n$ and $a|n$, so $\frac{n}{a}$ is an integer and $a\cdot\frac{n}{a}\equiv0\pmod n$.

The backwards implication is slightly trickier. 

When $n=2$, the result is trivial, so only odd primes are considered.

For every $a<n$, a unique multiplicative inverse, $a^{-1}$, exists. 

If $a\equiv a^{-1}\pmod n$, then $a^2\equiv 1\pmod n$, and $(a+1)(a-1)\equiv 0\pmod n$. Since $a<n$ and $n$ is prime, the only way this is possible is when $a=1$ or $a=n-1$, thus all numbers, between $2$ and $n-2$(inclusive) have a different unique multiplicative inverse.

The factors of $(p-2)!$ all results in $1\pmod n$. Multiplying this by $(p-1)$, we get $(p-1)!\equiv-1\pmod p$. Therefore the backwards implication is proven.

\begin{theorem}
	Fermat little theorem
	
	If $p$ is a prime number, then$a^p\equiv a\pmod p$. Alternatively, $a^{p-1}\equiv1\pmod p$ if $a\perp p$
\end{theorem}
The idea behind this is that there are $p-1$ elements in $\mathbb{Z}_p^\times$, since $p$ is prime, every integer except $0$ satisfies the group axioms. A simple proof involves considering the sequence

$$a,2a,3a,\dots,(p-1)a$$
This sequence is simply a rearrangement of 
$$1,2,3,\dots,p-1$$
if $a\neq 0\pmod p$(simple proof by contradiction)

Now we consider the product of both sequences
$$a^{p-1}(p-1)!\equiv(p-1)!\pmod p$$
By Wilson's theorem,
$$-a^{p-1}\equiv-1\pmod p$$
$$a^{p-1}\equiv1\pmod p$$
And the case where $a\equiv0\pmod p$ is trivial.

Notice this only works for primes. There exists a more general theorem that works for all integers:
\begin{theorem}
	Euler's theorem
	
	If $n\perp a$, then $$a^{\phi(n)}\equiv1\pmod n$$
\end{theorem}
Here $\phi(n)$ is the Euler's totient function.

//proof	

//Carmichael function/theorem

\section{Lattices(groups)}

A lattice is a discrete subgroup of $\left(\mathbb{R}^+\right)^n$. It is also quite clear that a lattice can be constructed from a set of 'basis' vectors, $B=\left\{b_i\right\}_{i=0}^n$, and the lattice is simply $$\left\{\left.\sum_{k=0}^{n}x_ib_i\right|x_i\in\mathbb{Z}\right\}$$

This is used in many places of mathematics and applied mathematics, however we will focus on lattice problems in computational theory.


//lll

//using lll in weird ways
\chapter{Abstract algebra}
Abstract algebra is a branch of mathematics that focus on the structure of objects. Such objects include groups, fields, algebras, and many more.

Recommended readings:

\begin{itemize}
	\item I. N. Herstein - Topics in algebra (a more older look into algebra)
	\item N. Jacobson - Basic algebra (a more modern algebra book)
\end{itemize}

\section{Group}

A group is a set $G$, along with a operator $\cdot$ defined by $4$ axioms:

\begin{itemize}
	\item If $a,b\in G$, then $a\cdot b\in G$
	\item $(a\cdot b)\cdot c=a\cdot(b\cdot c)$
	\item There exists an element $e\in G$ such that $e\cdot a=a=a\cdot e$
	\item For every $a\in G$, there exists an element $a^{-1}$ such that $a\cdot a^{-1}=e$
\end{itemize}

A few properties come directly from this axioms:

\begin{corollary}
	$e$ is unique
\end{corollary}

\begin{corollary}
	$$a\cdot a^{-1}=a^{-1}\cdot a$$
	Note that it isn't necessary that $a\cdot b=b\cdot a$
\end{corollary}

If $a\cdot b=b\cdot a$ for all $a,b\in G$, this is known as a \textbf{abelian group}

The \textbf{order} of a group is simply the number of elements in it. We denote the order of a group $G$ as $o(G)$, alternatively using a similar notation to the cardinality of a set, $|G|$

\subsection{Subgroups and cosets}

A \textbf{subgroup} is simply a subset of the group that is a group, for example, the the even integers is a subgroup of the integers under addition.

Suppose we have a subgroup $H$ of $G$, if we select an element $a\notin H$, we define the sets:
$$aH=\{ah|h\in H\}$$
$$Ha=\{ha|h\in H\}$$
as the \textbf{left and right cosets} respectively. 'coset' will usually refer to the right coset

\begin{corollary}
	$|aH|=|Ha|=|H|$
\end{corollary}

\begin{corollary}
	(Left) cosets are either identical or completely disjoint 
\end{corollary}

Now we can proceed to proceed to proof one of the most important theorems in group theory, Lagrange theorem.

\begin{theorem}
	Let $H$ be a subgroup of $G$, then $o(H)|o(G)$
\end{theorem}

Suppose there are $k$ distinct cosets of $H$.

Since $e$ is in $H$, the union of all the cosets must be $G$.

All cosets are also either distinct or identical, so the union of all $k$ distinct cosets must be $G$.

Since the cosets are distinct, the size is simply the sum of the individual cosets, therefore
$$ko(H)=o(G)$$

\subsection{Morphisms}

Homomorphism ...
Endomorphism ...

Epimorphism ...
Monomorphism ...
Isomorphism ...
Automorphism ...

\subsection{Generators, abelian and cyclic groups}

\textbf{Generators} ...

\textbf{Abelian groups} are simply groups where elements are commutative, that means $ab=ba$ for all $a,b\in G$. For these kind of groups, we have a special notation:

\begin{center}
	\begin{tabular}{|c|c|}\hline
		General&Abelian\\\hline
		$ab$&$a+b$\\\hline
		$a^n$&$na$\\\hline
	\end{tabular}
\end{center}

This gives the idea of `commutative-ness' as addition is usually seen as commutative($a+b=b+a$) while multiplication is not quite commutative since something as simple as matrix multiplication is already non-commutative.

\textbf{Cyclic groups} are groups that are generated by one single element. 

//cyclic is abelian

//isomorphic cyclic

//order of cyclic group basically nth root lol

//$\exp G=|G|$ then $G$ is cyclic

\part{Theoretical concepts}
//we dont really care for actual attacks tbh...
\part{Block ciphers}
\chapter{Schemes}
\section{S and P boxes}
\section{Feistel network}
used everywhere
\section{Lai Massey network}
\chapter{Hashes}
hashes are kinda blockciphery
\chapter{DES}
\chapter{AES}
\chapter{Alternate block ciphers}
\section{ChaCha}
used by google apparently?
\part{Classical public-key cryptosystems}
\chapter{Old ciphers}
//just describe general techniques going through all is quite rarted
\chapter{Key exchange}
\text{Key exchange} are methods to share a private key over a public channel such that any evedropper cannot recover the private key with similar computing powers.

\section{Merkel secret gifts}

\section{Diffie Hellman}

\textbf{Diffie Hellman} is a way of sharing a `random' value by communication in a open public channel

The algorithm:

\begin{enumerate}
	\item Choose a prime $p$ and a generator $p>g\neq0,1$
	\item Alice selects a private key $a$ and Bob selects a private key $b$
	\item Alice shares $g^a\pmod p$ and Bob shares $g^b\pmod p$
	\item Alice and Bob computes $g^{ab}\pmod p$, the shared private key
\end{enumerate}

The difficulty of a evedropper calculating $g^{ab}\pmod p$ is solving the discrete log problem - finding $a$ given $g^a\pmod p$. This is trivial under integers or reals but nontrivial for finite abelian groups. This algorithm can be used for any abelian finite groups to get a shared private key.

\subsection{Elliptic curve}
A similar method can be used replacing integers mod $p$ with the elliptic curve discrete group:

\section{Attacks}

\begin{enumerate}
	\item BSGS
	\item Pohlig-Hellman
	\item Logjam(precomputing NFS)
\end{enumerate}

\begin{enumerate}
	\item Choose a prime $p$ and a curve $C$, and a point $P\neq\mathcal{O}$
	\item Alice selects a private key $a$ and Bob selects a private key $b$
	\item Alice shares $aP$ and Bob shares $bP$
	\item Alice and Bob computes $sbP$, the shared private key
\end{enumerate}

\chapter{RSA}
\section{Introduction}

RSA is a private key cryptography algorithm that relies on prime factorization being hard.

The algorithm:

\begin{enumerate}
	\item Generate $2$ distinct primes, $p$ and $q$
	\item Let $n=pq$
	\item Compute either $o=\phi(n)$ or $\lambda(n)$
	\item Choose $1<e<o$, such that $\gcd(e,o)=0$
	\item Calculate $d$ such that $ed=1\pmod o$
	\item Public: $n,e$\\Private: $p,q,o,d$
	\item To encrypt $m$, calculate $c=m^e\mod n$
	\item To decrypt $c$, calculate $m=c^d\mod n$
\end{enumerate}

\section{Variants}
\subsection{Multiple primes}
This is rather simple, just use multiple primes for $n$, the rest of the algorithm works, its possible that a prime repeats, so just make sure to use the correct $\phi(n)$ since $\phi\left(p^2\right)=p(p-1)\neq(p-1)^2$.
\subsection{Speed up}
Sometimes to speed up decryption, CRT is used to compute $\tmod n$.

Let $d_p=d\pmod{p-1}$, $d_q=d\pmod{q-1}$
$$m_p\equiv c^d\equiv c^{d_p}\pmod p$$
$$m_q\equiv c^d\equiv c^{d_q}\pmod q$$
Define $h$ as $m=m_q+qh$(by CRT). Now consider $m\pmod p$.
$$m_q+qh\equiv m_p\pmod p$$
$$h\equiv q^{-1}\left(m_p-m_q\right)\pmod p$$
\subsection{$e=2$}
For decryption modulo square root will be needed, followed by trying out all possible signs since positive and negative solutions are possible. After that use chinese remainder theorem.
\section{Attacks}
There are quite a few of attacks on RSA if the system is improperly setup:
\begin{itemize}
	\item Small $n$/Week $p$
	\item Small $d$
	\item Small $p-q$
	\item Partial $d$ exposure
	\item Partial $p$ exposure
	\item Partial $m$ exposure
	\item Partial decryption oracle(LSB kind) oracle
	\item Padding oracle
	\item Constant $n,m$ different $e$
	\item Constant $e,m$ different $n$
	\item Constant $e$, related $m$, different $n$
	\item Timing attack
	\item Power trace
	\item Fault attack
\end{itemize}

\subsection{Important papers}:

\begin{itemize}
	\item https://crypto.stanford.edu/~dabo/papers/RSA-survey.pdf contains most of the attacks on RSA
\end{itemize}

\subsection{Small $n$}
To factor $n$, I suggest using yafu, Alpertron online factorization or looking up on factordb(may also work for larger primes)
\subsection{Small $d$}
This attack is called the \textbf{Wiener attack}.

The idea of this attack is that we can use continued fractions to find a good approximation for $\frac{e}{n}$, and one of the approximations will be off the form $\frac{k}{d}$ where $k$ is a integer.

Criteria: Let $n=pq$, then if $p<q<2p$ and $d<\frac{\sqrt[4]{n}}{3}$, $d$ can be recovered easily from $e$.

Since $ed = 1\pmod{\lambda(n)}$, there exists a $k$ such that $ed-k\lambda(n)=1$. Therefore


$$\frac{e}{\lambda(n)}-\frac{k}{d}=\frac{1}{d \lambda(n)}$$

Let $g=\gcd(p-1,q-1)$, then

$$\frac{e}{\phi(n)}-\frac{k}{gd}=\frac{1}{d\phi(n)}$$ 

From this $\frac{k}{Gd}$ is an approximation of $\frac{e}{\phi(n)}$, so a continued fraction method could be used to obtain $\frac{k}{Gd}$. However the attacker does not know $\phi(n)$, instead, $n$ can be used to approximate $\phi(n)$. Since
$\phi(n)=n-p-q+1$ and by assumption $p+q-1<3\sqrt{n}$
$$n-\phi(n)<3\sqrt{n}$$

Using $n$ instead of $\phi(n)$ we obtain:


$$\frac{e}{n}-\frac{k}{gd}=\frac{edg-kn}{ngd}$$
$$=\frac{edg-k\phi(n)-kn+k\phi(n)}{ngd}$$

$$= \left \vert \frac{1-k(n-\phi(n))}{ngd} \right \vert $$

$$\le\frac{3k\sqrt{n}}{ngd}=\frac{3k\sqrt{n}}{\sqrt{n}\sqrt{n}gd} \le\frac{3k}{d\sqrt{n}}$$

Since $k \lambda(n)=ed-1<ed$, $k \lambda(n)<ed<\lambda(n)d$, thus $k<d$ and by assumption $d<\frac{1}{3}\sqrt[4]{n}$.

Hence:

$$\frac{e}{n}-\frac{k}{gd}\le\frac{1}{d\sqrt[4]{n}}$$

Since $d<\frac{1}{3}\sqrt[4]{n}$, $2d<3d<\sqrt[4]{n}$, we get

$$\frac{e}{n}-\frac{k}{gd}\le\frac{3k}{d\sqrt{n}}<\frac{1}{2d^2}$$

If $ \left\vert x-\frac{a}{b} \right\vert<\frac{1}{2b^2}$, then $\frac{a}{b}$ is a convergent of $x$, thus $\frac{k}{d}$ appears among the convergents of $\frac{e}{n}$.

We have done many approximations in this proof, so it seems like the bound could be improved. Using continued fractions this is the best bound known so far, however other methods allows us to push the bound to $d<N^{0.292}$ and maybe a few extra bits

https://sci-hub.tw/10.1007/3-540-48910-X\_1

https://www.ncbi.nlm.nih.gov/pmc/articles/PMC3985315/


\subsection{Small $p-q$}
The basic attack for this is \textbf{Fermat factorization}.

Notice that $$(a+b)(a-b)=a^2-b^2$$

Thus if we can find $a,b$ such that $a^2-N=b^2$, we have factored $N$. 

\begin{lstlisting}
def fermat(n):
	a=ceil(sqrt(n))
	b2=a*a-n
	while !is_square(b2):
		a+=1
		b2=a*a-n
	b=sqrt(b2)
	return (a-b,a+b)
\end{lstlisting}

There are several other speedups to this method.

https://eprint.iacr.org/2009/318.pdf
\subsection{Partial $d$ exposure}
//coppersmith lll
\subsection{Partial $p$ exposure}
//coppersmith lll
\subsection{Partial $m$ exposure}
//coppersmith small roots
\subsection{Partial decryption oracle(LSB kind) oracle}
//multiply by $k^e$ and abuse modulo
\subsection{Padding oracle}
//multiply by $k^e$ and abuse modulo

//Bleichenbacher
\subsection{Constant $n,m$ different $e$}
Suppose we are given $c_1=m^{e_1}\pmod n$ and $c_2=m^{e_2}\pmod n$. 

Using B\'ezout's identity, we compute $a,b$ such that $ae_1+be_2=1$, then we compute $$c_1^ac_2^b=m^{ae_1}m^{be_2}=m\pmod n$$

\subsection{Constant $e,m$ different $n$}

The idea of this comes from when $m$ is small, we can easily take the $e$th root.

For example, $c=6369690780153,e=3,n=25160293800283$, we can take the cube root of $c$ and get $m=0\text{x}4869=$`Hi'. If we were given multiple $m^e mod n_i$, we can `increase' the modulus using chinese remainder theorem

Suppose we receive $c_i=m^e\pmod n_i$ for $e$ such $i$, we can simply use CRT to find $c=m^e\pmod \Pi_i n_i$, and take the $e$th root of $c$, since $m<n$, so $m^e\lessapprox\Pi_in_i$.

\subsection{Constant $e$, related $m$, different $n$}
//ok now have magic with Hastad or FR atks
\subsection{Timing attack}
//https://www.paulkocher.com/TimingAttacks.pdf

//http://crypto.stanford.edu/~dabo/papers/ssl-timing.pdf
\subsection{Power trace}
actually just 0 and 1
\subsection{Fault attack}
lol just f up one of the crt modulos ez
\chapter{ECC}
algebraic geometry hell here we go
\part{Quantum cryptography}
\chapter{Basic theory of qubits}
Qubits are like normal bits, but probabilistic and much harder to imagine. We represent qubits as probability density, as compared to classical probability.

\section{States}

We represent states with \textbf{ket vectors}. This notation will become clear once more quantum theory is introduced.

For a given state $A$, we call the state $\ket{A}$. Qubits have $2$ states, $0$ and $1$, $\ket{0}$ and $\ket{1}$ respectively.

Classically we think of probability as a real positive number less than $1$, however for qubits, it's easier to think in terms of \textbf{probability density}, which can be complex. The probability of an event occurring is given by the square of the probability density.

Basic properties of probability density:
\begin{enumerate}
	\item A complex number with magnitude less than $1$
	\item The chance is given by the squared magnitude of the probability density
	\item Squared sum is $1$(square then sum)
\end{enumerate}

A general qubit, $\ket{\psi}$ has some chance of being $\ket{0}$ and some chance of being $\ket{1}$, we represent this as 
$$\ket{\psi}=\alpha\ket{0}+\beta\ket{1}$$
where $P(\ket{0})=|\alpha|^2$, $P(\ket{1})=|\beta|^2$

Now rewriting this into the more familiar vector notation, we can represent $\ket{0}$ as $\begin{pmatrix}1\\0\end{pmatrix}$ and $\ket{1}$ as $\begin{pmatrix}0\\1\end{pmatrix}$, so a arbitrary quantum state is written as
$$\ket{\psi}=\begin{pmatrix}\alpha\\\beta\end{pmatrix}$$

Now we introduce the dual of the ket notation, the \textbf{bra vector}. These forms a bra-c-ket when placed together. The bra is written as the opposite of a ket, $\bra{0}$, and it's defined as(using vectors)

$$\bra{\psi}=\ket{\psi}^\dagger$$

where $\dagger$ represents conjugate transpose(complex conjugate and transpose)

So $\bra{\psi}=\begin{pmatrix}\alpha^*&\beta^*\end{pmatrix}$, and when we place them together, we get

$$\braket{\psi}{\psi}=\begin{pmatrix}\alpha^*&\beta^*\end{pmatrix}\begin{pmatrix}\alpha\\\beta\end{pmatrix}=|\alpha|^2+|\beta|^2=1$$

//projecting stuff with braket

//multiqubit state

//entanglement ayy

//chsh with the funky integration proof

\chapter{BB84}
\chapter{E91}
\part{Post-quantum crypto}
fancy math
\end{document}